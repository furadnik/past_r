% Options for packages loaded elsewhere
\PassOptionsToPackage{unicode}{hyperref}
\PassOptionsToPackage{hyphens}{url}
%
\documentclass[
]{article}
\usepackage{amsmath,amssymb}
\usepackage{iftex}
\ifPDFTeX
  \usepackage[T1]{fontenc}
  \usepackage[utf8]{inputenc}
  \usepackage{textcomp} % provide euro and other symbols
\else % if luatex or xetex
  \usepackage{unicode-math} % this also loads fontspec
  \defaultfontfeatures{Scale=MatchLowercase}
  \defaultfontfeatures[\rmfamily]{Ligatures=TeX,Scale=1}
\fi
\usepackage{lmodern}
\ifPDFTeX\else
  % xetex/luatex font selection
\fi
% Use upquote if available, for straight quotes in verbatim environments
\IfFileExists{upquote.sty}{\usepackage{upquote}}{}
\IfFileExists{microtype.sty}{% use microtype if available
  \usepackage[]{microtype}
  \UseMicrotypeSet[protrusion]{basicmath} % disable protrusion for tt fonts
}{}
\makeatletter
\@ifundefined{KOMAClassName}{% if non-KOMA class
  \IfFileExists{parskip.sty}{%
    \usepackage{parskip}
  }{% else
    \setlength{\parindent}{0pt}
    \setlength{\parskip}{6pt plus 2pt minus 1pt}}
}{% if KOMA class
  \KOMAoptions{parskip=half}}
\makeatother
\usepackage{xcolor}
\usepackage[margin=1in]{geometry}
\usepackage{color}
\usepackage{fancyvrb}
\newcommand{\VerbBar}{|}
\newcommand{\VERB}{\Verb[commandchars=\\\{\}]}
\DefineVerbatimEnvironment{Highlighting}{Verbatim}{commandchars=\\\{\}}
% Add ',fontsize=\small' for more characters per line
\usepackage{framed}
\definecolor{shadecolor}{RGB}{248,248,248}
\newenvironment{Shaded}{\begin{snugshade}}{\end{snugshade}}
\newcommand{\AlertTok}[1]{\textcolor[rgb]{0.94,0.16,0.16}{#1}}
\newcommand{\AnnotationTok}[1]{\textcolor[rgb]{0.56,0.35,0.01}{\textbf{\textit{#1}}}}
\newcommand{\AttributeTok}[1]{\textcolor[rgb]{0.13,0.29,0.53}{#1}}
\newcommand{\BaseNTok}[1]{\textcolor[rgb]{0.00,0.00,0.81}{#1}}
\newcommand{\BuiltInTok}[1]{#1}
\newcommand{\CharTok}[1]{\textcolor[rgb]{0.31,0.60,0.02}{#1}}
\newcommand{\CommentTok}[1]{\textcolor[rgb]{0.56,0.35,0.01}{\textit{#1}}}
\newcommand{\CommentVarTok}[1]{\textcolor[rgb]{0.56,0.35,0.01}{\textbf{\textit{#1}}}}
\newcommand{\ConstantTok}[1]{\textcolor[rgb]{0.56,0.35,0.01}{#1}}
\newcommand{\ControlFlowTok}[1]{\textcolor[rgb]{0.13,0.29,0.53}{\textbf{#1}}}
\newcommand{\DataTypeTok}[1]{\textcolor[rgb]{0.13,0.29,0.53}{#1}}
\newcommand{\DecValTok}[1]{\textcolor[rgb]{0.00,0.00,0.81}{#1}}
\newcommand{\DocumentationTok}[1]{\textcolor[rgb]{0.56,0.35,0.01}{\textbf{\textit{#1}}}}
\newcommand{\ErrorTok}[1]{\textcolor[rgb]{0.64,0.00,0.00}{\textbf{#1}}}
\newcommand{\ExtensionTok}[1]{#1}
\newcommand{\FloatTok}[1]{\textcolor[rgb]{0.00,0.00,0.81}{#1}}
\newcommand{\FunctionTok}[1]{\textcolor[rgb]{0.13,0.29,0.53}{\textbf{#1}}}
\newcommand{\ImportTok}[1]{#1}
\newcommand{\InformationTok}[1]{\textcolor[rgb]{0.56,0.35,0.01}{\textbf{\textit{#1}}}}
\newcommand{\KeywordTok}[1]{\textcolor[rgb]{0.13,0.29,0.53}{\textbf{#1}}}
\newcommand{\NormalTok}[1]{#1}
\newcommand{\OperatorTok}[1]{\textcolor[rgb]{0.81,0.36,0.00}{\textbf{#1}}}
\newcommand{\OtherTok}[1]{\textcolor[rgb]{0.56,0.35,0.01}{#1}}
\newcommand{\PreprocessorTok}[1]{\textcolor[rgb]{0.56,0.35,0.01}{\textit{#1}}}
\newcommand{\RegionMarkerTok}[1]{#1}
\newcommand{\SpecialCharTok}[1]{\textcolor[rgb]{0.81,0.36,0.00}{\textbf{#1}}}
\newcommand{\SpecialStringTok}[1]{\textcolor[rgb]{0.31,0.60,0.02}{#1}}
\newcommand{\StringTok}[1]{\textcolor[rgb]{0.31,0.60,0.02}{#1}}
\newcommand{\VariableTok}[1]{\textcolor[rgb]{0.00,0.00,0.00}{#1}}
\newcommand{\VerbatimStringTok}[1]{\textcolor[rgb]{0.31,0.60,0.02}{#1}}
\newcommand{\WarningTok}[1]{\textcolor[rgb]{0.56,0.35,0.01}{\textbf{\textit{#1}}}}
\usepackage{graphicx}
\makeatletter
\def\maxwidth{\ifdim\Gin@nat@width>\linewidth\linewidth\else\Gin@nat@width\fi}
\def\maxheight{\ifdim\Gin@nat@height>\textheight\textheight\else\Gin@nat@height\fi}
\makeatother
% Scale images if necessary, so that they will not overflow the page
% margins by default, and it is still possible to overwrite the defaults
% using explicit options in \includegraphics[width, height, ...]{}
\setkeys{Gin}{width=\maxwidth,height=\maxheight,keepaspectratio}
% Set default figure placement to htbp
\makeatletter
\def\fps@figure{htbp}
\makeatother
\setlength{\emergencystretch}{3em} % prevent overfull lines
\providecommand{\tightlist}{%
  \setlength{\itemsep}{0pt}\setlength{\parskip}{0pt}}
\setcounter{secnumdepth}{-\maxdimen} % remove section numbering
\ifLuaTeX
  \usepackage{selnolig}  % disable illegal ligatures
\fi
\IfFileExists{bookmark.sty}{\usepackage{bookmark}}{\usepackage{hyperref}}
\IfFileExists{xurl.sty}{\usepackage{xurl}}{} % add URL line breaks if available
\urlstyle{same}
\hypersetup{
  pdftitle={Zákony velkých čísel a CLV},
  hidelinks,
  pdfcreator={LaTeX via pandoc}}

\title{Zákony velkých čísel a CLV}
\author{}
\date{\vspace{-2.5em}}

\begin{document}
\maketitle

\hypertarget{ilustrace-zux103ux2c7kona-velkux103ch-uxe4ux165ux103sel}{%
\section{Ilustrace zákona velkých
ÄŤĂ­sel}\label{ilustrace-zux103ux2c7kona-velkux103ch-uxe4ux165ux103sel}}

Sice mluví o limitě, ale vidíme, že konvergence je tady zjevná už
pro relativně malá \(n\). Černá čára znázorňuje posloupnost
průměrů, tj. \(S_1, S_2, \dots\)

\begin{Shaded}
\begin{Highlighting}[]
\NormalTok{n }\OtherTok{=} \DecValTok{10}\SpecialCharTok{\^{}}\DecValTok{4}
\NormalTok{X }\OtherTok{=} \FunctionTok{runif}\NormalTok{(n)}
\NormalTok{S }\OtherTok{=} \FunctionTok{cumsum}\NormalTok{(X)}\SpecialCharTok{/}\NormalTok{(}\DecValTok{1}\SpecialCharTok{:}\NormalTok{n)}
\FunctionTok{plot}\NormalTok{(S, }\AttributeTok{type=}\StringTok{\textquotesingle{}l\textquotesingle{}}\NormalTok{)}
\FunctionTok{curve}\NormalTok{(}\FloatTok{0.5}\SpecialCharTok{+}\DecValTok{0}\SpecialCharTok{*}\NormalTok{x,}\AttributeTok{from=}\DecValTok{0}\NormalTok{,}\AttributeTok{to=}\NormalTok{n,}\AttributeTok{col=}\StringTok{\textquotesingle{}red\textquotesingle{}}\NormalTok{,}\AttributeTok{add=}\NormalTok{T)}
\end{Highlighting}
\end{Shaded}

\includegraphics{main_files/figure-latex/unnamed-chunk-1-1.pdf}

Znázornění téhož pomocí histogramu -- průměr je silně
koncentrovaný okolo střední hodnoty.

\begin{Shaded}
\begin{Highlighting}[]
\NormalTok{SZVC }\OtherTok{=} \ControlFlowTok{function}\NormalTok{ (n, show) \{}
\NormalTok{  N }\OtherTok{=} \DecValTok{10}\SpecialCharTok{\^{}}\DecValTok{4}
\NormalTok{  m }\OtherTok{=} \FunctionTok{matrix}\NormalTok{(}\FunctionTok{runif}\NormalTok{(n}\SpecialCharTok{*}\NormalTok{N), }\AttributeTok{nrow=}\NormalTok{N)     }\CommentTok{\# matice nezávislých náhodných veličin}
\NormalTok{  Y }\OtherTok{=}\NormalTok{ (}\FunctionTok{rowSums}\NormalTok{(m)}\SpecialCharTok{{-}}\NormalTok{n}\SpecialCharTok{/}\DecValTok{2}\NormalTok{)}\SpecialCharTok{/}\NormalTok{(n) }\CommentTok{\# každá položka Y vznikne sečtením a "přeškálováním" jednoho řádku m}
  \ControlFlowTok{if}\NormalTok{ (}\SpecialCharTok{!}\NormalTok{show) \{ }\FunctionTok{pdf}\NormalTok{(}\AttributeTok{file=}\FunctionTok{paste}\NormalTok{(}\StringTok{"SZVC\_unif{-}"}\NormalTok{, n, }\StringTok{".pdf"}\NormalTok{, }\AttributeTok{sep=}\StringTok{""}\NormalTok{)); \}}
  \FunctionTok{curve}\NormalTok{(dnorm, }\AttributeTok{main=}\FunctionTok{paste}\NormalTok{(}\StringTok{"X\_i je U(0,1), n="}\NormalTok{, n), }\AttributeTok{ylim=}\FunctionTok{c}\NormalTok{(}\DecValTok{0}\NormalTok{,}\DecValTok{5}\NormalTok{), }\AttributeTok{from=}\SpecialCharTok{{-}}\DecValTok{2}\NormalTok{, }\AttributeTok{to=}\DecValTok{2}\NormalTok{, }\AttributeTok{col=}\StringTok{"red"}\NormalTok{)        }\CommentTok{\# pro srování hustota normálního rozdělení }
  \FunctionTok{hist}\NormalTok{(Y,}\AttributeTok{breaks=}\FunctionTok{seq}\NormalTok{(}\SpecialCharTok{{-}}\DecValTok{8}\NormalTok{,}\DecValTok{8}\NormalTok{,}\AttributeTok{by=}\FloatTok{0.1}\NormalTok{), }\AttributeTok{freq =} \ConstantTok{FALSE}\NormalTok{, }\AttributeTok{add=}\ConstantTok{TRUE}\NormalTok{) }\CommentTok{\# a do něj nakreslený histogram Y}
  \FunctionTok{legend}\NormalTok{(}\StringTok{"topright"}\NormalTok{,}\AttributeTok{lty=}\DecValTok{1}\NormalTok{,}\AttributeTok{lwd=}\DecValTok{3}\NormalTok{,}\AttributeTok{col=}\FunctionTok{c}\NormalTok{(}\StringTok{\textquotesingle{}black\textquotesingle{}}\NormalTok{, }\StringTok{\textquotesingle{}red\textquotesingle{}}\NormalTok{),}
       \AttributeTok{legend=}\FunctionTok{c}\NormalTok{(}\StringTok{"Y\_n"}\NormalTok{, }\StringTok{"N(0,1)"}\NormalTok{),}\AttributeTok{bty=}\StringTok{"n"}\NormalTok{)    }
  \ControlFlowTok{if}\NormalTok{ (}\SpecialCharTok{!}\NormalTok{show) \{ }\FunctionTok{dev.off}\NormalTok{(); \}}
\NormalTok{\}}

\FunctionTok{SZVC}\NormalTok{(}\DecValTok{10}\NormalTok{,T)}
\end{Highlighting}
\end{Shaded}

\includegraphics{main_files/figure-latex/unnamed-chunk-2-1.pdf}

\begin{Shaded}
\begin{Highlighting}[]
\FunctionTok{SZVC}\NormalTok{(}\DecValTok{200}\NormalTok{,T)}
\end{Highlighting}
\end{Shaded}

\includegraphics{main_files/figure-latex/unnamed-chunk-2-2.pdf}

\hypertarget{ilustrace-centrux103ux2c7lnux103-limitnux103-vuxe4ty}{%
\section{Ilustrace centrální limitní
věty}\label{ilustrace-centrux103ux2c7lnux103-limitnux103-vuxe4ty}}

Při přeškálování ``odmocninou'' dochází stále k výrazné
oscilací -- distribuce \(Y_n\) (na obrázku hodnota v bodě \(x=n\)) se
blíží normálnímu rozdělení, tj. bude v průměru vzdálena od 0 o 1,
i pro obrovská \(n\). Oproti tomu distribuce \(S_n\) ze zákona
velkých čísel se blíží jednobodovému rozdělení na hodnotě \(\mu\).

\begin{Shaded}
\begin{Highlighting}[]
\NormalTok{n }\OtherTok{=} \DecValTok{10}\SpecialCharTok{\^{}}\DecValTok{5}
\NormalTok{X }\OtherTok{=} \FunctionTok{runif}\NormalTok{(n)}
\NormalTok{Y }\OtherTok{=}\NormalTok{ (}\FunctionTok{cumsum}\NormalTok{(X)}\SpecialCharTok{{-}}\NormalTok{(}\DecValTok{1}\SpecialCharTok{:}\NormalTok{n)}\SpecialCharTok{/}\DecValTok{2}\NormalTok{)}\SpecialCharTok{/}\NormalTok{(}\FunctionTok{sqrt}\NormalTok{(}\DecValTok{1}\SpecialCharTok{/}\DecValTok{12}\NormalTok{)}\SpecialCharTok{*}\FunctionTok{sqrt}\NormalTok{(}\DecValTok{1}\SpecialCharTok{:}\NormalTok{n))}
\FunctionTok{plot}\NormalTok{(Y, }\AttributeTok{type=}\StringTok{\textquotesingle{}l\textquotesingle{}}\NormalTok{)}
\FunctionTok{curve}\NormalTok{(}\DecValTok{0}\SpecialCharTok{+}\DecValTok{0}\SpecialCharTok{*}\NormalTok{x,}\AttributeTok{from=}\DecValTok{0}\NormalTok{,}\AttributeTok{to=}\NormalTok{n,}\AttributeTok{col=}\StringTok{\textquotesingle{}red\textquotesingle{}}\NormalTok{,}\AttributeTok{add=}\NormalTok{T)}
\end{Highlighting}
\end{Shaded}

\includegraphics{main_files/figure-latex/unnamed-chunk-3-1.pdf}

\hypertarget{vsuvka-z-r-kovux103ho-programovux103ux2c7nux103}{%
\section{Vsuvka z R-kového
programování}\label{vsuvka-z-r-kovux103ho-programovux103ux2c7nux103}}

Dva způsoby jak udělat kumulovanou sumu

\begin{Shaded}
\begin{Highlighting}[]
\NormalTok{X }\OtherTok{=} \FunctionTok{rep}\NormalTok{(}\DecValTok{1}\NormalTok{,}\DecValTok{10}\NormalTok{); X}
\end{Highlighting}
\end{Shaded}

\begin{verbatim}
##  [1] 1 1 1 1 1 1 1 1 1 1
\end{verbatim}

\begin{Shaded}
\begin{Highlighting}[]
\NormalTok{Y }\OtherTok{=} \FunctionTok{cumsum}\NormalTok{(X); Y}
\end{Highlighting}
\end{Shaded}

\begin{verbatim}
##  [1]  1  2  3  4  5  6  7  8  9 10
\end{verbatim}

\begin{Shaded}
\begin{Highlighting}[]
\FunctionTok{Reduce}\NormalTok{(}\StringTok{"+"}\NormalTok{, X, }\AttributeTok{accumulate=}\NormalTok{T)}
\end{Highlighting}
\end{Shaded}

\begin{verbatim}
##  [1]  1  2  3  4  5  6  7  8  9 10
\end{verbatim}

\hypertarget{histogramy-y_n-pro-x_i-sim-u01}{%
\section{\texorpdfstring{Histogramy \(Y_n\) pro
\(X_i \sim U(0,1)\)}{Histogramy Y\_n pro X\_i \textbackslash sim U(0,1)}}\label{histogramy-y_n-pro-x_i-sim-u01}}

\begin{Shaded}
\begin{Highlighting}[]
\NormalTok{CLV1 }\OtherTok{=} \ControlFlowTok{function}\NormalTok{ (n, }\AttributeTok{show=}\ConstantTok{FALSE}\NormalTok{) \{}
\NormalTok{  N }\OtherTok{=} \DecValTok{10}\SpecialCharTok{\^{}}\DecValTok{6}
\NormalTok{  m }\OtherTok{=} \FunctionTok{matrix}\NormalTok{(}\FunctionTok{runif}\NormalTok{(n}\SpecialCharTok{*}\NormalTok{N, }\DecValTok{0}\NormalTok{,}\DecValTok{1}\NormalTok{), }\AttributeTok{nrow=}\NormalTok{N)     }\CommentTok{\# matice nezávislých náhodných veličin U(0,1)}
\NormalTok{  Y }\OtherTok{=}\NormalTok{ (}\FunctionTok{rowSums}\NormalTok{(m)}\SpecialCharTok{{-}}\NormalTok{n}\SpecialCharTok{/}\DecValTok{2}\NormalTok{)}\SpecialCharTok{/}\NormalTok{(}\FunctionTok{sqrt}\NormalTok{(}\DecValTok{1}\SpecialCharTok{/}\DecValTok{12}\NormalTok{)}\SpecialCharTok{*}\FunctionTok{sqrt}\NormalTok{(n)) }\CommentTok{\# každá položka Y vznikne sečtením a "přeškálováním" jednoho řádku m}
  \ControlFlowTok{if}\NormalTok{ (}\SpecialCharTok{!}\NormalTok{show)  \{ }
\CommentTok{\#    pdf(file=paste("CLV\_unif{-}", toString(n), ".pdf", sep=""))}
    \FunctionTok{pdf}\NormalTok{(}\AttributeTok{file=}\FunctionTok{paste}\NormalTok{(}\StringTok{"CLV\_en\_unif{-}"}\NormalTok{, }\FunctionTok{toString}\NormalTok{(n), }\StringTok{".pdf"}\NormalTok{, }\AttributeTok{sep=}\StringTok{""}\NormalTok{))    }
\NormalTok{  \}}
\CommentTok{\#  curve(dnorm, main=paste("X\_i je U(0,1), n=",n), ylim=c(0,1), from={-}3, to=3, col=\textquotesingle{}red\textquotesingle{})        \# pro srování hustota normálního rozdělení }
  \FunctionTok{hist}\NormalTok{(Y, }\AttributeTok{breaks=}\FunctionTok{seq}\NormalTok{(}\SpecialCharTok{{-}}\DecValTok{5}\NormalTok{,}\DecValTok{5}\NormalTok{,}\AttributeTok{by=}\FloatTok{0.1}\NormalTok{), }\AttributeTok{freq =} \ConstantTok{FALSE}\NormalTok{, }\AttributeTok{ylim=}\FunctionTok{c}\NormalTok{(}\DecValTok{0}\NormalTok{,.}\DecValTok{5}\NormalTok{), }\AttributeTok{add=}\ConstantTok{FALSE}\NormalTok{) }\CommentTok{\# a do něj nakreslený histogram Y}
  \FunctionTok{curve}\NormalTok{(dnorm, }\AttributeTok{main=}\FunctionTok{paste}\NormalTok{(}\StringTok{"X\_i is U(0,1), n="}\NormalTok{,n), }\AttributeTok{ylim=}\FunctionTok{c}\NormalTok{(}\DecValTok{0}\NormalTok{,.}\DecValTok{5}\NormalTok{), }\AttributeTok{from=}\SpecialCharTok{{-}}\DecValTok{3}\NormalTok{, }\AttributeTok{to=}\DecValTok{3}\NormalTok{, }\AttributeTok{col=}\StringTok{\textquotesingle{}red\textquotesingle{}}\NormalTok{, }\AttributeTok{add=}\NormalTok{T)        }\CommentTok{\# pro srování hustota normálního rozdělení   }
  
  \FunctionTok{legend}\NormalTok{(}\StringTok{"topright"}\NormalTok{,}\AttributeTok{lty=}\DecValTok{1}\NormalTok{,}\AttributeTok{lwd=}\DecValTok{3}\NormalTok{,}\AttributeTok{col=}\FunctionTok{c}\NormalTok{(}\StringTok{\textquotesingle{}black\textquotesingle{}}\NormalTok{, }\StringTok{\textquotesingle{}red\textquotesingle{}}\NormalTok{),}
       \AttributeTok{legend=}\FunctionTok{c}\NormalTok{(}\StringTok{"Y\_n"}\NormalTok{, }\StringTok{"N(0,1)"}\NormalTok{),}\AttributeTok{bty=}\StringTok{"n"}\NormalTok{)  }
  \ControlFlowTok{if}\NormalTok{ (}\SpecialCharTok{!}\NormalTok{show) \{ }\FunctionTok{dev.off}\NormalTok{()\}}
\NormalTok{\}}
\end{Highlighting}
\end{Shaded}

\begin{Shaded}
\begin{Highlighting}[]
\ControlFlowTok{for}\NormalTok{ (n }\ControlFlowTok{in} \DecValTok{1}\SpecialCharTok{:}\DecValTok{10}\NormalTok{) \{ }\FunctionTok{CLV1}\NormalTok{(n,T); \}}
\end{Highlighting}
\end{Shaded}

\includegraphics{main_files/figure-latex/unnamed-chunk-6-1.pdf}
\includegraphics{main_files/figure-latex/unnamed-chunk-6-2.pdf}
\includegraphics{main_files/figure-latex/unnamed-chunk-6-3.pdf}
\includegraphics{main_files/figure-latex/unnamed-chunk-6-4.pdf}
\includegraphics{main_files/figure-latex/unnamed-chunk-6-5.pdf}
\includegraphics{main_files/figure-latex/unnamed-chunk-6-6.pdf}
\includegraphics{main_files/figure-latex/unnamed-chunk-6-7.pdf}
\includegraphics{main_files/figure-latex/unnamed-chunk-6-8.pdf}
\includegraphics{main_files/figure-latex/unnamed-chunk-6-9.pdf}
\includegraphics{main_files/figure-latex/unnamed-chunk-6-10.pdf} \#
Histogramy \(Y_n\) pro \(X_i \sim U(0,1)\)

\begin{Shaded}
\begin{Highlighting}[]
\NormalTok{CLV2 }\OtherTok{=} \ControlFlowTok{function}\NormalTok{ (n, }\AttributeTok{show=}\ConstantTok{FALSE}\NormalTok{) \{}
  \CommentTok{\# N je počet pokusů použitých pro samplování histogramů Y\_n}
  \CommentTok{\# n je index Y\_n, tj. kolik n.n.v. sčítám}
\NormalTok{  N }\OtherTok{=} \DecValTok{10}\SpecialCharTok{\^{}}\DecValTok{6}
\NormalTok{  m }\OtherTok{=} \FunctionTok{matrix}\NormalTok{(}\FunctionTok{rexp}\NormalTok{(n}\SpecialCharTok{*}\NormalTok{N, }\DecValTok{1}\NormalTok{), }\AttributeTok{nrow=}\NormalTok{N)     }\CommentTok{\# matice nezávislých náhodných veličin}
\NormalTok{  Y }\OtherTok{=}\NormalTok{ (}\FunctionTok{rowSums}\NormalTok{(m)}\SpecialCharTok{{-}}\NormalTok{n)}\SpecialCharTok{/}\NormalTok{(}\DecValTok{1}\SpecialCharTok{/}\FunctionTok{sqrt}\NormalTok{(}\DecValTok{1}\NormalTok{)}\SpecialCharTok{*}\FunctionTok{sqrt}\NormalTok{(n)) }\CommentTok{\# každá položka Y vznikne sečtením a "přeškálováním" jednoho řádku m}
  \ControlFlowTok{if}\NormalTok{ (}\SpecialCharTok{!}\NormalTok{show)  \{ }
\CommentTok{\#    pdf(file=paste("CLV\_exp{-}", toString(n), ".pdf", sep=""))}
    \FunctionTok{pdf}\NormalTok{(}\AttributeTok{file=}\FunctionTok{paste}\NormalTok{(}\StringTok{"CLV\_en\_exp{-}"}\NormalTok{, }\FunctionTok{toString}\NormalTok{(n), }\StringTok{".pdf"}\NormalTok{, }\AttributeTok{sep=}\StringTok{""}\NormalTok{))    }
\NormalTok{  \}}
  \FunctionTok{hist}\NormalTok{(Y, }\AttributeTok{breaks=}\FunctionTok{seq}\NormalTok{(}\SpecialCharTok{{-}}\DecValTok{5}\NormalTok{,}\DecValTok{20}\NormalTok{,}\AttributeTok{by=}\FloatTok{0.1}\NormalTok{), }\AttributeTok{freq =} \ConstantTok{FALSE}\NormalTok{, }\AttributeTok{ylim=}\FunctionTok{c}\NormalTok{(}\DecValTok{0}\NormalTok{,.}\DecValTok{5}\NormalTok{), }\AttributeTok{add=}\ConstantTok{FALSE}\NormalTok{) }\CommentTok{\# a do něj nakreslený histogram Y}
  \FunctionTok{curve}\NormalTok{(dnorm, }\AttributeTok{main=}\FunctionTok{paste}\NormalTok{(}\StringTok{"X\_i is Exp(1), n="}\NormalTok{,n), }\AttributeTok{ylim=}\FunctionTok{c}\NormalTok{(}\DecValTok{0}\NormalTok{,.}\DecValTok{5}\NormalTok{), }\AttributeTok{from=}\SpecialCharTok{{-}}\DecValTok{3}\NormalTok{, }\AttributeTok{to=}\DecValTok{3}\NormalTok{, }\AttributeTok{col=}\StringTok{\textquotesingle{}red\textquotesingle{}}\NormalTok{, }\AttributeTok{add=}\NormalTok{T)        }\CommentTok{\# pro srování hustota normálního rozdělení   }
\CommentTok{\#  curve(dnorm, main=paste("X\_i je Exp(1), n=",n), ylim=c(0,1), from={-}3, to=3, col=\textquotesingle{}red\textquotesingle{})        \# pro srování hustota normálního rozdělení }
\CommentTok{\#hist(Y, breaks=seq({-}20,20,by=0.2), freq = FALSE, add=TRUE) \# a do něj nakreslený histogram Y}
  \FunctionTok{legend}\NormalTok{(}\StringTok{"topright"}\NormalTok{,}\AttributeTok{lty=}\DecValTok{1}\NormalTok{,}\AttributeTok{lwd=}\DecValTok{3}\NormalTok{,}\AttributeTok{col=}\FunctionTok{c}\NormalTok{(}\StringTok{\textquotesingle{}black\textquotesingle{}}\NormalTok{, }\StringTok{\textquotesingle{}red\textquotesingle{}}\NormalTok{),}
       \AttributeTok{legend=}\FunctionTok{c}\NormalTok{(}\StringTok{"Y\_n"}\NormalTok{, }\StringTok{"N(0,1)"}\NormalTok{),}\AttributeTok{bty=}\StringTok{"n"}\NormalTok{)  }
  \ControlFlowTok{if}\NormalTok{ (}\SpecialCharTok{!}\NormalTok{show) \{ }\FunctionTok{dev.off}\NormalTok{()\}}
\NormalTok{\}}
\end{Highlighting}
\end{Shaded}

\begin{Shaded}
\begin{Highlighting}[]
\ControlFlowTok{for}\NormalTok{ (n }\ControlFlowTok{in} \DecValTok{1}\SpecialCharTok{:}\DecValTok{20}\NormalTok{) \{ }\FunctionTok{CLV2}\NormalTok{(n,T); \}}
\end{Highlighting}
\end{Shaded}

\includegraphics{main_files/figure-latex/unnamed-chunk-8-1.pdf}
\includegraphics{main_files/figure-latex/unnamed-chunk-8-2.pdf}
\includegraphics{main_files/figure-latex/unnamed-chunk-8-3.pdf}
\includegraphics{main_files/figure-latex/unnamed-chunk-8-4.pdf}
\includegraphics{main_files/figure-latex/unnamed-chunk-8-5.pdf}
\includegraphics{main_files/figure-latex/unnamed-chunk-8-6.pdf}
\includegraphics{main_files/figure-latex/unnamed-chunk-8-7.pdf}
\includegraphics{main_files/figure-latex/unnamed-chunk-8-8.pdf}
\includegraphics{main_files/figure-latex/unnamed-chunk-8-9.pdf}
\includegraphics{main_files/figure-latex/unnamed-chunk-8-10.pdf}
\includegraphics{main_files/figure-latex/unnamed-chunk-8-11.pdf}
\includegraphics{main_files/figure-latex/unnamed-chunk-8-12.pdf}
\includegraphics{main_files/figure-latex/unnamed-chunk-8-13.pdf}
\includegraphics{main_files/figure-latex/unnamed-chunk-8-14.pdf}
\includegraphics{main_files/figure-latex/unnamed-chunk-8-15.pdf}
\includegraphics{main_files/figure-latex/unnamed-chunk-8-16.pdf}
\includegraphics{main_files/figure-latex/unnamed-chunk-8-17.pdf}
\includegraphics{main_files/figure-latex/unnamed-chunk-8-18.pdf}
\includegraphics{main_files/figure-latex/unnamed-chunk-8-19.pdf}
\includegraphics{main_files/figure-latex/unnamed-chunk-8-20.pdf}

\end{document}
